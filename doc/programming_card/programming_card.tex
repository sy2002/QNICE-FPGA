\documentclass{leaflet}
\usepackage{longtable}
\begin{document}
 \title{QNICE programming card}
 \maketitle
%
 \section{General}
  QNICE is a 16 bit processor featuring four addressing modes, 16 registers and
  a 16 bit address space of 16 bit words, the upper 1 kW page is reserved for
  memory mapped I/O.
%
 \section{Registers}
  All in all there are 16 general purpose registers (\emph{GPR}s) available:
  \begin{center}
   \begin{longtable}{|c|c|c||c|c|c|c|c|}
    \hline
    {\tt R0}&\dots&{\tt R7}&{\tt R8}&\dots&{\tt R13}&{\tt R14}&{\tt R15}\\
    \hline
   \end{longtable}
  \end{center}
  \begin{description}
   \item [\texttt{R0}\dots\texttt{R7}:] General purpose registers, actually 
    these are a window into a register bank holding $256\times 8$ such 
    registers.
   \item [\texttt{R13}:] Stack pointer (\texttt{SP}).
   \item [\texttt{R14}:] Statusregister (\texttt{SR}).
   \item [\texttt{R15}:] Program counter (\texttt{PC}).
  \end{description}
%
  \subsection{\texttt{SR}}
   \begin{center}
    \begin{longtable}{|c||c|c|c|c|c|c|c|c|}
     \hline
     {\tt rbank}&
     {\tt M}&{\tt I}&{\tt V}&{\tt N}&{\tt Z}&{\tt C}&{\tt X}&{\tt 1}\\
     \hline
    \end{longtable}
   \end{center}
    \begin{description}
     \item [{\tt 1}:] Always set to 1
     \item [{\tt X}:] 1 if the last result was {\tt 0xFFFF}
     \item [{\tt C}:] Carry flag
     \item [{\tt Z}:] 1 if the last result was {\tt 0x0000}
     \item [{\tt N}:] 1 if the last result was negative
     \item [{\tt V}:] 1 if the last operation caused an overflow
     \item [{\tt I}:] 1 if an interrupt occured
     \item [{\tt M}:] If set to 1, maskable interrupts are allowed
    \end{description}
    The upper eight bits of \texttt{SR} hold the pointer to the register
    window. Changing the value stored here will yield a different set of 
    GPRs \texttt{R0}\dots\texttt{R7} which is especially useful for subroutine
    calls.
%
 \section{Instruction Set}
  QNICE features 14 basic instructions, four jump/branch instructions, and 
  four adressing modes.
%
  \subsection{Basic Instructions}
   \begin{center}
    \begin{longtable}{|c||c|c||c|c|}
     \hline
     4 bit&4 bit&2 bit&4 bit&2 bit\\
     {\tt opcode}&{\tt src rxx}&{\tt src mode}&
     {\tt dst rxx}&{\tt dst mode}\\
     \hline
    \end{longtable}
   \end{center}
   {\scriptsize
    \begin{center}
     \begin{longtable}{|c|ll|l|}
      \hline
       Opc&Instr&Operands&Effect\\
      \hline
       {\tt 0}&{\tt MOVE}&{\tt src, dst}&{\tt dst := src}\\
       {\tt 1}&{\tt ADD}&{\tt src, dst}&{\tt dst := dst + src}\\
       {\tt 2}&{\tt ADDC}&{\tt src, dst}&{\tt dst := dst + src + C}\\
       {\tt 3}&{\tt SUB}&{\tt src, dst}&{\tt dst := dst - src}\\
       {\tt 4}&{\tt SUBC}&{\tt src, dst}&{\tt dst := dst - src - C}\\
       {\tt 5}&{\tt SHL}&{\tt src, dst}&{\tt dst << src}, fill with X, shift to C\\
       {\tt 6}&{\tt SHR}&{\tt src, dst}&{\tt dst >> src}, fill with C, shift to X\\
       {\tt 7}&{\tt SWAP}&{\tt src, dst}&{\tt dst := ((src << 8) \& 0xFF00) |}\\
              &          &              &{\tt ((src >> 8) \& 0xFF)}\\
       {\tt 8}&{\tt NOT}&{\tt src, dst}&{\tt dst := !src}\\
       {\tt 9}&{\tt AND}&{\tt src, dst}&{\tt dst := dst \& src}\\
       {\tt A}&{\tt OR}&{\tt src, dst}&{\tt dst := dst | src}\\
       {\tt B}&{\tt XOR}&{\tt src, dst}&{\tt dst := dst \^\ src}\\
       {\tt C}&{\tt CMP}&{\tt src, dst}&compare {\tt src} with {\tt dst}\\
       {\tt D}&&&reserved\\
       {\tt E}&{\tt HALT}&&Halt the processor\\
       {\tt D}&{\tt ABRA}&{\tt dest, [!]cond}&Absolute branch\\
       {\tt D}&{\tt ASUB}&{\tt dest, [!]cond}&Absolut subroutine call\\
       {\tt D}&{\tt RBRA}&{\tt dest, [!]cond}&Relative branch\\
       {\tt D}&{\tt RSUB}&{\tt dest, [!]cond}&Relative subroutine call\\
      \hline
     \end{longtable}
    \end{center}
   }
%
  \subsection{Jumps and Branches}
   {\scriptsize
    \begin{center}
     \begin{longtable}{|c||c|c||c||c|c|}
      \hline
      4 bit&4 bit&2 bit&2 bit&1 bit&3 bit\\
      &     &     &     &{\tt negate}&{\tt select}\\
      {\tt opcode}&{\tt src rxx}&{\tt src mode}&
      {\tt mode}&{\tt condition}&{\tt condition}\\
      \hline
     \end{longtable}
    \end{center}
   }
%
  \subsection{Addressing Modes}
   {\scriptsize
    \begin{center}
     \begin{longtable}{|c|l|l|}
      \hline
       Mode bits&Notation&Description\\
      \hline
       {\tt 00}&{\tt Rxx}&Use Rxx as operand\\
       {\tt 01}&{\tt @Rxx}&Use the memory cell addressed by\\
               &          &the contents of Rxx as operand\\
       {\tt 10}&{\tt @Rxx++}&Use the memory cell addressed by\\
               &          &the contents of Rxx as operand and\\
               &          &then increment Rxx\\
       {\tt 11}&{\tt @--Rxx}&Decrement Rxx and then use the\\
               &          &memory cell addressed by Rxx as\\
               &          &operand\\
      \hline
     \end{longtable}
    \end{center}
   }
%
  \subsection{Shortcuts}
   The file \texttt{sysdef.asm} (part of the monitor) defines some shortcuts
   which facilitate write- and readability of QNICE assembler code:
   \begin{center}
    \begin{longtable}{|l|l|}
     \hline
      Shortcut&Implementation\\
     \hline
      \texttt{RET}&\texttt{MOVE @R13++, R15}\\
      \texttt{INCRB}&\texttt{ADD 0x0100, R14}\\
      \texttt{DECRB}&\texttt{SUB 0x0100, R14}\\
      \texttt{NOP}&\texttt{ABRA R15, 1}\\
      \texttt{SYSCALL(x, y)}&\texttt{ASUB x, y}\\
     \hline
    \end{longtable}
   \end{center}
   \texttt{sysdef.asm} also defines three shortcuts \texttt{SP}, \texttt{SR},
   and \texttt{PC} for \texttt{R13}, \texttt{R14}, and \texttt{R15}.
%
 \section{Input/Output}
  I/O devices are memory mapped, their respective control and data registers 
  occupy the topmost 1 kW memory page.
\pagebreak
  {\scriptsize
   \begin{center}
    \begin{longtable}{|l|l|l|}
     \hline
     Label&Address&Description\\
     \hline
     \hline
     \texttt{IO\$BASE}&\texttt{0xFF00}&Start of I/O area\\
     \hline
     \texttt{VGA\$STATE}&\texttt{0xFF00}&VGA status register\\
     \texttt{VGA\$CR\_X}&\texttt{0xFF01}&Cursor X-position\\
     \texttt{VGA\$CR\_Y}&\texttt{0xFF02}&Cursor y-position\\
     \texttt{VGA\$CHAR}&\texttt{0xFF03}&Character code\\
     \texttt{VGA\$OFFS\_DISPLAY}&\texttt{0xFF04}&Display RAM offset\\
     \texttt{VGA\$OFFS\_RW}&\texttt{0xFF05}&R/W RAM offset\\
     \hline
     \texttt{IO\$TIL\_DISPLAY}&\texttt{0xFF10}&TIL-display\\
     \texttt{IO\$TIL\_MASK}&\texttt{0xFF11}&Mask register\\
     \hline
     \texttt{IO\$SWITCH\_REG}&\texttt{0xFF12}&Switch register\\
     \hline
     \texttt{IO\$KBD\_STATE}&\texttt{0xFF13}&USB-keyboard state\\
     \texttt{IO\$KBD\_DATA}&\texttt{0xFF14}&USB-keyboard data\\
     \hline
     \texttt{IO\$CYC\_LO}&\texttt{0xFF17}&Cycle counter low\\
     \texttt{IO\$CYC\_MID}&\texttt{0xFF18}&Cycle counter middle\\
     \texttt{IO\$CYC\_HI}&\texttt{0xFF19}&Cycle counter high\\
     \texttt{IO\$CYC\_STATE}&\texttt{0xFF1A}&Cycle counter status\\
     \hline
     \texttt{IO\$UART\_SRA}&\texttt{0xFF21}&UART status register\\
     \texttt{IO\$UART\_RHRA}&\texttt{0xFF22}&UART receive register\\
     \texttt{IO\$UART\_THRA}&\texttt{0xFF23}&UART receive register\\
     \hline
    \end{longtable}
   \end{center}
  }
%
  \subsection{VGA Controller}
   \subsubsection{\texttt{VGA\$STATE Bits}}
    \begin{center}
     \begin{longtable}{|l|l|}
      \hline
      Bit&Description\\
      \hline
      11&Enable R/W offset register if set.\\
      10&Enable display offset register if set.\\
      9&Busy (wait for 0 before issuing command).\\
      8&Clear screen (set until completion).\\
      7&Enable VGA controller.\\
      6&Enable hardware cursor.\\
      5&Enable hardware cursor blinking.\\
      4&Hardware cursor mode:\\
       &Small if set, large if cleared.\\
      2\dots 0&Display color (RGB).\\
      \hline
     \end{longtable}
    \end{center}
   \subsubsection{\texttt{VGA\$CR\_X}}
    Set this register to the X coordinate for the next character to be 
    displayed.
   \subsubsection{\texttt{VGA\$CR\_Y}}
    Y coordinate for the next character to be displayed.
   \subsubsection{\texttt{VGA\$CHAR}}
    Writing a byte to this register causes it to be displayed on the current
    X/Y coordinate on the screen. Reading from this register yields the 
    character at the current display coordinate.
   \subsubsection{\texttt{VGA\$OFFS\_DISPLAY}}
    This register holds the offset in bytes that is to be used when displaying
    the video RAM. To scroll one line forward, simply add \texttt{0x0050} to 
    this register. For this to work, bit 10 in \texttt{VGA\$STATE} has to be
    set.
   \subsubsection{\texttt{VGA\$OFFS\_RW}}
    Similar to \texttt{VGA\$OFFS\_DISPLAY} -- controls the offset for 
    read/write accesses to the display memory.
  \subsection{USB-Keyboard}
   \subsubsection{\texttt{IO\$KBD\_STATE}}
    \begin{center}
     \begin{longtable}{|l|l|}
      \hline
       Bit&Description\\
      \hline
       0&Set if an unread character is available.\\
       1&Function/cursor/\dots key pressed.\\
        &The value is stored in bits \texttt{15\dots8}.\\
       2\dots 4&Keyboard layout:\\
        &\texttt{000}: US keyboard\\
        &\texttt{001}: German keyboard\\
       5\dots 7&Key modifier bit mask:\\
        &5: shift, 6: alt, 7: ctrl\\
      \hline
     \end{longtable}
    \end{center}
   \subsection{Cycle Counter}
    \subsubsection{\texttt{CYC\$STATE}}
     \begin{center}
      \begin{longtable}{|l|l|}
       \hline
        Bit&Description\\
       \hline
        0&Reset counter and start counting.\\
        1&1: count, 0: inhibit\\
       \hline
      \end{longtable}
     \end{center}
%
  \subsection{UART}
   \subsubsection{\texttt{IO\$UART\_SRA}}
    \begin{center}
     \begin{longtable}{|l|l|}
      \hline
       Bit&Description\\
      \hline
       0&Character received.\\
       1&Transmitter ready for next character.\\
      \hline
     \end{longtable}
    \end{center}
%
 \section{Code Examples}
  \subsection{Typical Subroutine Call}
   \begin{verbatim}
        MOVE ..., R8     ; Setup subroutine 
                         ; parameters
        ...
        RSUB SUBR, 1     ; Call subroutine
        ...
SUBR:   ADD 0x0100, R14  ; Get free lower
                         ; register set
        ...
        SUB 0x0100, R14  ; Restore lower
                         ; register bank
        MOVE @R13++, R15 ; RET
   \end{verbatim}
  \subsection{Compute $\sum_{i=0}{\texttt{0x0010}}$}
   \begin{verbatim}
        .ORG 0x8000
        XOR R0, R0      ; Clear R0
        MOVE 0x0010, R1 ; Upper limit
LOOP:   ADD R1, R0      ; One summation
        SUB 0x0001, R1  ; Decrement i
        ABRA LOOP, !Z   ; Loop if not zero
        HALT
   \end{verbatim}
\end{document}
